\documentclass[12pt,]{article}
\usepackage{lmodern}
\usepackage{amssymb,amsmath}
\usepackage{ifxetex,ifluatex}
\usepackage{fixltx2e} % provides \textsubscript
\ifnum 0\ifxetex 1\fi\ifluatex 1\fi=0 % if pdftex
  \usepackage[T1]{fontenc}
  \usepackage[utf8]{inputenc}
\else % if luatex or xelatex
  \ifxetex
    \usepackage{mathspec}
    \usepackage{xltxtra,xunicode}
  \else
    \usepackage{fontspec}
  \fi
  \defaultfontfeatures{Mapping=tex-text,Scale=MatchLowercase}
  \newcommand{\euro}{€}
\fi
% use upquote if available, for straight quotes in verbatim environments
\IfFileExists{upquote.sty}{\usepackage{upquote}}{}
% use microtype if available
\IfFileExists{microtype.sty}{%
\usepackage{microtype}
\UseMicrotypeSet[protrusion]{basicmath} % disable protrusion for tt fonts
}{}
\usepackage[margin=1in]{geometry}
\usepackage{graphicx}
\makeatletter
\def\maxwidth{\ifdim\Gin@nat@width>\linewidth\linewidth\else\Gin@nat@width\fi}
\def\maxheight{\ifdim\Gin@nat@height>\textheight\textheight\else\Gin@nat@height\fi}
\makeatother
% Scale images if necessary, so that they will not overflow the page
% margins by default, and it is still possible to overwrite the defaults
% using explicit options in \includegraphics[width, height, ...]{}
\setkeys{Gin}{width=\maxwidth,height=\maxheight,keepaspectratio}
\ifxetex
  \usepackage[setpagesize=false, % page size defined by xetex
              unicode=false, % unicode breaks when used with xetex
              xetex]{hyperref}
\else
  \usepackage[unicode=true]{hyperref}
\fi
\hypersetup{breaklinks=true,
            bookmarks=true,
            pdfauthor={},
            pdftitle={Is the Tea Party Libertarian, Authoritarian, or Something Else?},
            colorlinks=true,
            citecolor=black,
            urlcolor=red,
            linkcolor=black,
            pdfborder={0 0 0}}
\urlstyle{same}  % don't use monospace font for urls
\setlength{\parindent}{0pt}
\setlength{\parskip}{6pt plus 2pt minus 1pt}
\setlength{\emergencystretch}{3em}  % prevent overfull lines
\setcounter{secnumdepth}{0}

%%% Change title format to be more compact
\usepackage{titling}
\setlength{\droptitle}{-2em}
  \title{Supplementary Information for ``Is the Tea Party Libertarian, Authoritarian, or Something Else?''
\vspace{1.25em}}
  \pretitle{\vspace{\droptitle}\centering\huge}
  \posttitle{\par}
  \author{}
  \preauthor{}\postauthor{}
  \date{}
  \predate{}\postdate{}


\usepackage{setspace}


\begin{document}

\maketitle

\paragraph{Contents}\label{contents}

\begin{enumerate}
\def\labelenumi{\arabic{enumi}.}
\itemsep1pt\parskip0pt\parsep0pt
\item
Additional information on Factor Analysis
\item
  Scree plot for factor model
\item
  Numerical factor loadings
\item
  Including Party ID as a covariate
\item
  Estimated effects of misarchism on Party ID and Conservatism
\item
  Additional information on Bayesian Model Averaging
\item
  Inclusion probabilities and expected value of coefficients from Bayesian Model Averaging
\item
  Additional information on Multiple Imputation
\item
  Pooled regression results after Multiple Imputation
\item
  Multiple Imputation diagnostics
\item
Additional information on matching estimates and sensitivity bounds
\item
References
\end{enumerate}

Additional information on Factor Analysis

We used an oblique rotation which allows for factors to be correlated. This is
  appropriate here because, while we argue that moral statism and
  anti-governmentalism are unique and distinct, they are likely to be
  correlated. We use maximum likelihood as the factoring method because
  it has a more formal statistical basis than other methods and is
  widely seen as one of the optimal methods. See Fabrigar et al. (1999). We estimate two factors, which a scree plot (below) suggests to be the optimal number.
  
  For \emph{N} equal to 5914, a chi-square test of the hypothesis that two factors is
  sufficient is equal to 900.727 with a p-value of 0, suggesting that
  two factors are not sufficient, as we would expect. That said, the
  root mean square of the residuals is 0.044 and the Tucker-Lewis Index
  for factor reliability score is 0.869, both of which are near the
  conventional cutoffs of .05 and .9, respectively.

\begin{figure}[htbp]
\centering
\includegraphics{figures/scree-1.pdf}
\caption{Scree plot shows elbow at two factors}
\end{figure}

\clearpage

\begin{table}[ht]
\centering
\begin{tabular}{rrr}
  \hline
 & Moral Statism & Governmentalism \\ 
  \hline
Conservatism & 0.56 & -0.26 \\ 
  Family & 0.64 & 0.08 \\ 
  GunControl & -0.06 & 0.39 \\ 
  Intolerant & 0.42 & -0.12 \\ 
  Morals & 0.35 & -0.21 \\ 
  Wiretapping & 0.37 & 0.24 \\ 
  DefenseSpending & 0.54 & 0.08 \\ 
  Services & -0.00 & 0.77 \\ 
  ImmigrationChecks & 0.39 & -0.21 \\ 
  JobGuarantee & -0.06 & 0.60 \\ 
   \hline
\end{tabular}
\caption{Factor Loadings} 
\label{Factor Loadings}
\end{table}

\clearpage

\begin{table}[!htbp] \centering 
  \caption{Alternative Dependent Variables for Model 2} 
  \label{} 
\footnotesize 
\begin{tabular}{@{\extracolsep{5pt}}lcc} 
\\[-1.8ex]\hline 
\hline \\[-1.8ex] 
 & \multicolumn{2}{c}{\textit{Dependent variable:}} \\ 
\cline{2-3} 
\\[-1.8ex] & PartyID (Republican) & Conservatism \\ 
\\[-1.8ex] & (1) & (2)\\ 
\hline \\[-1.8ex] 
 Gender (Male) & $-$0.005 & 0.005 \\ 
  & (0.013) & (0.011) \\ 
  Income & 0.033$^{**}$ & $-$0.002 \\ 
  & (0.015) & (0.013) \\ 
  Age & $-$0.068$^{***}$ & $-$0.029$^{**}$ \\ 
  & (0.014) & (0.012) \\ 
  Race (White) & 0.066$^{***}$ & $-$0.053$^{***}$ \\ 
  & (0.016) & (0.013) \\ 
  Education & 0.047$^{***}$ & 0.005 \\ 
  & (0.015) & (0.013) \\ 
  Obama & $-$0.549$^{***}$ & 0.068$^{***}$ \\ 
  & (0.020) & (0.020) \\ 
  Authoritarianism & $-$0.037$^{**}$ & $-$0.031$^{**}$ \\ 
  & (0.015) & (0.013) \\ 
  BornAgain & 0.014 & $-$0.007 \\ 
  & (0.014) & (0.012) \\ 
  Religion & 0.002 & 0.003 \\ 
  & (0.015) & (0.013) \\ 
  PartyID (Republican) &  & 0.172$^{***}$ \\ 
  &  & (0.017) \\ 
  FoxNews & 0.043$^{***}$ & 0.014 \\ 
  & (0.015) & (0.013) \\ 
  MoralStatism & 0.198$^{***}$ & 0.748$^{***}$ \\ 
  & (0.026) & (0.022) \\ 
  Government & $-$0.131$^{***}$ & $-$0.089$^{***}$ \\ 
  & (0.024) & (0.021) \\ 
  MoralStatism*Government & 0.022 & 0.060$^{***}$ \\ 
  & (0.027) & (0.023) \\ 
  Constant & $-$0.045$^{**}$ & 0.036$^{**}$ \\ 
  & (0.020) & (0.017) \\ 
 \hline \\[-1.8ex] 
Observations & 2,406 & 2,406 \\ 
R$^{2}$ & 0.661 & 0.712 \\ 
Adjusted R$^{2}$ & 0.659 & 0.710 \\ 
Residual Std. Error & 0.311 (df = 2392) & 0.266 (df = 2391) \\ 
F Statistic & 359.000$^{***}$ (df = 13; 2392) & 422.000$^{***}$ (df = 14; 2391) \\ 
\hline 
\hline \\[-1.8ex] 
\textit{Note:}  & \multicolumn{2}{r}{$^{*}$p$<$0.1; $^{**}$p$<$0.05; $^{***}$p$<$0.01} \\ 
\end{tabular} 
\end{table}

\begin{table}[!htbp] \centering 
  \caption{Including Party ID as a Covariate} 
  \label{} 
\footnotesize 
\begin{tabular}{@{\extracolsep{5pt}}lc} 
\\[-1.8ex]\hline 
\hline \\[-1.8ex] 
 & \multicolumn{1}{c}{\textit{Dependent variable:}} \\ 
\cline{2-2} 
\\[-1.8ex] & Tea Party Support \\ 
\hline \\[-1.8ex] 
 Gender (Male) & 0.132 \\ 
  & (0.129) \\ 
  Income & $-$0.322$^{**}$ \\ 
  & (0.148) \\ 
  Conservatism & 0.785$^{***}$ \\ 
  & (0.255) \\ 
  Age & $-$0.354$^{**}$ \\ 
  & (0.145) \\ 
  Race (White) & $-$0.419$^{**}$ \\ 
  & (0.168) \\ 
  Education & 0.057 \\ 
  & (0.159) \\ 
  Obama & $-$1.350$^{***}$ \\ 
  & (0.224) \\ 
  Authoritarianism & 0.052 \\ 
  & (0.148) \\ 
  BornAgain & 0.288$^{**}$ \\ 
  & (0.135) \\ 
  Religion & $-$0.027 \\ 
  & (0.156) \\ 
  PartyID (Republican) & 0.313 \\ 
  & (0.202) \\ 
  FoxNews & 0.682$^{***}$ \\ 
  & (0.134) \\ 
  MoralStatism & 0.093 \\ 
  & (0.337) \\ 
  Government & $-$0.670$^{**}$ \\ 
  & (0.265) \\ 
  MoralStatism*Government & $-$1.510$^{***}$ \\ 
  & (0.321) \\ 
  Constant & $-$2.570$^{***}$ \\ 
  & (0.208) \\ 
 \hline \\[-1.8ex] 
Observations & 2,406 \\ 
Log Likelihood & $-$828.000 \\ 
Akaike Inf. Crit. & 1,688.000 \\ 
\hline 
\hline \\[-1.8ex] 
\textit{Note:}  & \multicolumn{1}{r}{$^{*}$p$<$0.1; $^{**}$p$<$0.05; $^{***}$p$<$0.01} \\ 
\end{tabular} 
\end{table}

\clearpage

Additional information on Bayesian Model Averaging

A version of the R package \emph{BMA} modified by
Mongtomery and Nyhan was used to search over the entire model space of
Model 2. The analysis was conducted using the \emph{bic.glmMN()} function, which is a version of the
\emph{bic.glm()} function in the R package \emph{BMA}. See Raftery et
  al. (2015) and Montgomery and Nyhan (2010). The analysis uses
uniform priors for all independent variables and the only restriction is
that the interaction term and its two component variables are required
to enter or not enter models together. Options were set to be least restrictive.
No Occam's Window was used to narrow the models selected by the initial leap algorithm run over the entire model space and, following Montgomery and Nyhan (pp.~21), the number of best models of each size returned by the leaps algorithm was set to 100,000. In the end, 4,096 models were selected.

\begin{figure}[htbp]
\centering
\includegraphics{figures/bma2-1.pdf}
\caption{Inclusion Probabilites from Bayesian Model Averaging}
\end{figure}

\clearpage

\begin{figure}[htbp]
\centering
\includegraphics{figures/bma3-1.pdf}
\caption{Expected Values from Bayesian Model Averaging}
\end{figure}

\clearpage

Additional information on Multiple Imputation

We use the R package \emph{Amelia} (Honaker, King, and Blackwell 2008) to generate 10
versions of the ANES dataset with missing values imputed and
\emph{Zelig} (Imai, King, and Lau 2009) to obtain pooled regression results via ``Rubin's rules.'' The multiple
imputation algorithm only assumes that missing values are ``missing at
random,'' not necessarily ``missing completely at random.'' In this
context, ``missing at random'' only means that missingness is dependent
on the observed variables. Full numerical results are provided below. Graphical diagnostics for overimputation,
dispersion, and comparing pre- and post-imputation densities for our
main variables suggest no problems or anomolies in the imputation
procedures (see below the numerical results).

\begin{table}[ht]
\centering
\begin{tabular}{rrrrr}
  \hline
 & Value & Std. Error & t-stat & p-value \\ 
  \hline
(Intercept) & -2.08 & 0.12 & -18.04 & 0.00 \\ 
  Gender (Female) & -0.10 & 0.09 & -1.12 & 0.26 \\ 
  Income & -0.33 & 0.10 & -3.33 & 0.00 \\ 
  Age & -0.30 & 0.09 & -3.38 & 0.00 \\ 
  Race (White) & -0.30 & 0.11 & -2.68 & 0.01 \\ 
  Education & -0.04 & 0.10 & -0.43 & 0.66 \\ 
  Obama & -0.99 & 0.14 & -7.04 & 0.00 \\ 
  Authoritarianism & 0.06 & 0.11 & 0.60 & 0.55 \\ 
  BornAgain & -0.21 & 0.12 & -1.77 & 0.08 \\ 
  Religion & 0.02 & 0.09 & 0.19 & 0.85 \\ 
  PartyID (Republican) & 0.23 & 0.13 & 1.75 & 0.08 \\ 
  FoxNews & 0.66 & 0.10 & 6.90 & 0.00 \\ 
  MoralStatism & 0.98 & 0.18 & 5.35 & 0.00 \\ 
  Government & -0.67 & 0.17 & -4.03 & 0.00 \\ 
  MoralStatism*Government & -1.34 & 0.20 & -6.76 & 0.00 \\ 
   \hline
\end{tabular}
\caption{Pooled Logistic Regression Results From 10 Multiple Imputations} 
\end{table}

\clearpage

\begin{figure}[htbp]
\centering
\includegraphics{figures/missing2-1.pdf}
\caption{Distributions before and after multiple imputation}
\end{figure}

\begin{figure}[htbp]
\centering
\includegraphics{figures/missing3-1.pdf}
\caption{Distributions before and after multiple imputation}
\end{figure}

\clearpage

\begin{figure}[htbp]
\centering
\includegraphics{figures/missing4-1.pdf}
\caption{Diagnostic Plot for Overimputation (1)}
\end{figure}

\begin{figure}[htbp]
\centering
\includegraphics{figures/missing5-1.pdf}
\caption{Diagnostic Plot for Overimputation (2)}
\end{figure}

\clearpage

\begin{figure}[htbp]
\centering
\includegraphics{figures/missing6-1.pdf}
\caption{Diagnostic Plot for Dispersion}
\end{figure}

Additional information on matching estimates and sensitivity bounds

Rosenbaum bounds (1988) for a binary dependent variable are calculated using the \emph{binarysens()} function in the R package \emph{rbounds}. See Keele (2014). We generate matching estimates for each of our three independent
variables of interest, one at a time, using one-to-one genetic matching
with replacement. In each case, ``treatment'' is defined as having a
value above the sample mean of the variable of interest. For each
estimate, we balance on all covariates in Model 2 except the two
components of the interaction term and including the treatment
variable's propensity scores with respect to those covariates.

\clearpage

\subsection{References}

Fabrigar, Leandre R, Duane T Wegener, Robert C MacCallum, and Erin J
Strahan. 1999. ``Evaluating the use of exploratory factor analysis in
psychological research.'' \emph{Psychological Methods} 4(3): 272--99.

Keele, Luke J. 2014. ``rbounds: Perform Rosenbaum bounds sensitivity
tests for matched and unmatched data.''
\url{https://cran.r-project.org/web/packages/rbounds}.

Raftery, Adrian E., Jennifer Hoeting, Chris Volinksy, Ian Painter, and Ka Yee Yeung. 2015. ``BMA: Bayesian Model Averaging.''
\url{https://cran.r-project.org/web/packages/BMA/}.


\end{document}