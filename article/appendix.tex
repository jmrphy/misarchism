\documentclass[12pt,]{article}
\usepackage{lmodern}
\usepackage{amssymb,amsmath}
\usepackage{ifxetex,ifluatex}
\usepackage{fixltx2e} % provides \textsubscript
\ifnum 0\ifxetex 1\fi\ifluatex 1\fi=0 % if pdftex
  \usepackage[T1]{fontenc}
  \usepackage[utf8]{inputenc}
\else % if luatex or xelatex
  \ifxetex
    \usepackage{mathspec}
    \usepackage{xltxtra,xunicode}
  \else
    \usepackage{fontspec}
  \fi
  \defaultfontfeatures{Mapping=tex-text,Scale=MatchLowercase}
  \newcommand{\euro}{€}
\fi
% use upquote if available, for straight quotes in verbatim environments
\IfFileExists{upquote.sty}{\usepackage{upquote}}{}
% use microtype if available
\IfFileExists{microtype.sty}{%
\usepackage{microtype}
\UseMicrotypeSet[protrusion]{basicmath} % disable protrusion for tt fonts
}{}
\usepackage[margin=1in]{geometry}
\usepackage{graphicx}
\makeatletter
\def\maxwidth{\ifdim\Gin@nat@width>\linewidth\linewidth\else\Gin@nat@width\fi}
\def\maxheight{\ifdim\Gin@nat@height>\textheight\textheight\else\Gin@nat@height\fi}
\makeatother
% Scale images if necessary, so that they will not overflow the page
% margins by default, and it is still possible to overwrite the defaults
% using explicit options in \includegraphics[width, height, ...]{}
\setkeys{Gin}{width=\maxwidth,height=\maxheight,keepaspectratio}
\ifxetex
  \usepackage[setpagesize=false, % page size defined by xetex
              unicode=false, % unicode breaks when used with xetex
              xetex]{hyperref}
\else
  \usepackage[unicode=true]{hyperref}
\fi
\hypersetup{breaklinks=true,
            bookmarks=true,
            pdfauthor={},
            pdftitle={Is the Tea Party Libertarian, Authoritarian, or Something Else?},
            colorlinks=true,
            citecolor=black,
            urlcolor=red,
            linkcolor=black,
            pdfborder={0 0 0}}
\urlstyle{same}  % don't use monospace font for urls
\setlength{\parindent}{0pt}
\setlength{\parskip}{6pt plus 2pt minus 1pt}
\setlength{\emergencystretch}{3em}  % prevent overfull lines
\setcounter{secnumdepth}{0}

%%% Change title format to be more compact
\usepackage{titling}
\setlength{\droptitle}{-2em}
  \title{Supplementary Information for ``Is the Tea Party Libertarian, Authoritarian, or Something Else?''
\vspace{1.25em}}
  \pretitle{\vspace{\droptitle}\centering\huge}
  \posttitle{\par}
  \author{}
  \preauthor{}\postauthor{}
  \date{}
  \predate{}\postdate{}


\usepackage{setspace}


\begin{document}

\maketitle

\paragraph{Contents}\label{contents}

\begin{enumerate}
\def\labelenumi{\arabic{enumi}.}
\itemsep1pt\parskip0pt\parsep0pt
\item
Additional information on Factor Analysis
\item
  Scree plot for factor model
\item
  Numerical factor loadings
\item
  Including Party ID as a covariate
\item
  Estimated effects of misarchism on Party ID and Conservatism
\item
  Additional information on Bayesian Model Averaging
\item
  Inclusion probabilities and expected value of coefficients from Bayesian Model Averaging
\item
  Additional information on Multiple Imputation
\item
  Pooled regression results after Multiple Imputation
\item
  Multiple Imputation diagnostics
\item
Additional information on matching estimates and sensitivity bounds
\item
References
\end{enumerate}

Additional information on Factor Analysis

We used an oblique rotation which allows for factors to be correlated. This is
  appropriate here because, while we argue that moral statism and
  anti-governmentalism are unique and distinct, they are likely to be
  correlated. We use maximum likelihood as the factoring method because
  it has a more formal statistical basis than other methods and is
  widely seen as one of the optimal methods. See Fabrigar et al. (1999). We estimate two factors, which a scree plot (below) suggests to be the optimal number.
  
  For \emph{N} equal to 5914, a chi-square test of the hypothesis that two factors is
  sufficient is equal to 900.727 with a p-value of 0, suggesting that
  two factors are not sufficient, as we would expect. That said, the
  root mean square of the residuals is 0.044 and the Tucker-Lewis Index
  for factor reliability score is 0.869, both of which are near the
  conventional cutoffs of .05 and .9, respectively.

\begin{figure}[htbp]
\centering
\includegraphics{figures/scree-1.pdf}
\caption{Scree plot shows elbow at two factors}
\end{figure}

\clearpage

\begin{table}[ht]
\centering
\begin{tabular}{rrr}
  \hline
 & ML1 & ML2 \\ 
  \hline
Family & 0.03 & 0.59 \\ 
  Guns & 0.40 & -0.05 \\ 
  Intolerant & -0.12 & 0.44 \\ 
  Morals & -0.22 & 0.36 \\ 
  Wiretapping & 0.21 & 0.35 \\ 
  Defense & 0.04 & 0.49 \\ 
  Services & 0.78 & 0.01 \\ 
  Immigration & -0.24 & 0.38 \\ 
  Jobs & 0.61 & -0.04 \\ 
   \hline
\end{tabular}
\caption{Factor Loadings} 
\label{Factor Loadings}
\end{table}

\begin{figure}[htbp]
\centering
\includegraphics{figures/descriptives-1.pdf}
\caption{Correlation plot for different dimensions of right-wing
attitudes}
\end{figure}



\clearpage

\begin{table}[!htbp] \centering 
  \caption{Alternative Dependent Variables for Model 2} 
  \label{} 
\footnotesize 
\begin{tabular}{@{\extracolsep{5pt}}lcc} 
\\[-1.8ex]\hline 
\hline \\[-1.8ex] 
 & \multicolumn{2}{c}{\textit{Dependent variable:}} \\ 
\cline{2-3} 
\\[-1.8ex] & PartyID (Republican) & Conservatism \\ 
\\[-1.8ex] & (1) & (2)\\ 
\hline \\[-1.8ex] 
 Gender (Male) & $-$0.006 & 0.008 \\ 
  & (0.013) & (0.014) \\ 
  Income & 0.032$^{**}$ & $-$0.003 \\ 
  & (0.015) & (0.016) \\ 
  Age & $-$0.057$^{***}$ & 0.025 \\ 
  & (0.014) & (0.016) \\ 
  Race (White) & 0.076$^{***}$ & $-$0.058$^{***}$ \\ 
  & (0.015) & (0.017) \\ 
  Education & 0.044$^{***}$ & $-$0.024 \\ 
  & (0.015) & (0.017) \\ 
  Obama & $-$0.545$^{***}$ & $-$0.103$^{***}$ \\ 
  & (0.020) & (0.025) \\ 
  Authoritarianism & $-$0.029$^{**}$ & 0.039$^{**}$ \\ 
  & (0.015) & (0.016) \\ 
  BornAgain & 0.016 & 0.030$^{*}$ \\ 
  & (0.014) & (0.015) \\ 
  Religion & 0.001 & 0.038$^{**}$ \\ 
  & (0.015) & (0.017) \\ 
  PartyID (Republican) &  & 0.321$^{***}$ \\ 
  &  & (0.022) \\ 
  FoxNews & 0.039$^{**}$ & 0.059$^{***}$ \\ 
  & (0.015) & (0.017) \\ 
  Conservatism & 0.256$^{***}$ &  \\ 
  & (0.018) &  \\ 
  MoralStatism & $-$0.001 & 0.259$^{***}$ \\ 
  & (0.024) & (0.026) \\ 
  Government & $-$0.103$^{***}$ & $-$0.100$^{***}$ \\ 
  & (0.023) & (0.026) \\ 
  MoralStatism*Government & 0.025 & 0.044 \\ 
  & (0.026) & (0.029) \\ 
  Constant & $-$0.051$^{***}$ & $-$0.009 \\ 
  & (0.019) & (0.022) \\ 
 \hline \\[-1.8ex] 
Observations & 2,406 & 2,406 \\ 
R$^{2}$ & 0.674 & 0.523 \\ 
Adjusted R$^{2}$ & 0.673 & 0.521 \\ 
Residual Std. Error (df = 2391) & 0.305 & 0.342 \\ 
F Statistic (df = 14; 2391) & 354.000$^{***}$ & 187.000$^{***}$ \\ 
\hline 
\hline \\[-1.8ex] 
\textit{Note:}  & \multicolumn{2}{r}{$^{*}$p$<$0.1; $^{**}$p$<$0.05; $^{***}$p$<$0.01} \\ 
\end{tabular} 
\end{table}



\begin{table}[!htbp] \centering 
  \caption{Including ideology in factor model and removing from regression model} 
  \label{} 
\footnotesize 
\begin{tabular}{@{\extracolsep{5pt}}lc} 
\\[-1.8ex]\hline 
\hline \\[-1.8ex] 
 & \multicolumn{1}{c}{\textit{Dependent variable:}} \\ 
\cline{2-2} 
\\[-1.8ex] & Tea Party Support \\ 
\hline \\[-1.8ex] 
 Gender (Male) & 0.116 \\ 
  & (0.128) \\ 
  Income & $-$0.301$^{**}$ \\ 
  & (0.148) \\ 
  Age & $-$0.343$^{**}$ \\ 
  & (0.144) \\ 
  Race (White) & $-$0.404$^{**}$ \\ 
  & (0.169) \\ 
  Education & 0.066 \\ 
  & (0.158) \\ 
  Obama & $-$1.250$^{***}$ \\ 
  & (0.222) \\ 
  Authoritarianism & 0.035 \\ 
  & (0.148) \\ 
  BornAgain & 0.298$^{**}$ \\ 
  & (0.135) \\ 
  Religion & $-$0.013 \\ 
  & (0.155) \\ 
  PartyID (Republican) & 0.512$^{**}$ \\ 
  & (0.199) \\ 
  FoxNews & 0.707$^{***}$ \\ 
  & (0.133) \\ 
  MoralStatism & $-$0.556$^{**}$ \\ 
  & (0.261) \\ 
  Government & 0.923$^{***}$ \\ 
  & (0.276) \\ 
  MoralStatism*Government & 1.380$^{***}$ \\ 
  & (0.268) \\ 
  Constant & $-$2.700$^{***}$ \\ 
  & (0.214) \\ 
 \hline \\[-1.8ex] 
Observations & 2,406 \\ 
Log Likelihood & $-$830.000 \\ 
Akaike Inf. Crit. & 1,690.000 \\ 
\hline 
\hline \\[-1.8ex] 
\textit{Note:}  & \multicolumn{1}{r}{$^{*}$p$<$0.1; $^{**}$p$<$0.05; $^{***}$p$<$0.01} \\ 
\end{tabular} 
\end{table}

Individuals characterized by the highest observed levels of moral
statism also in the 90th percentile of anti-governmentalism (below a
value of -.7 on the governmentalism factor) have a probability of
supporting the Tea Party around 40\%. To be more precise, moving from
minimum to maxium moral statism while in the 10th percentile of
governmentalism increases the probability of supporting the Tea Party by
.39 (sd=.07) more than it would in the 90th percentile of
govermentalism, from .01 (sd=.00) to .42, (sd=.07) rather than .05
(sd=.02) to .03 (sd=.02).

\clearpage

Bayesian Model Averaging

A version of the R package \emph{BMA} modified by
Mongtomery and Nyhan was used to search over the entire model space of
Model 2. The analysis was conducted using the \emph{bic.glmMN()} function, which is a version of the
\emph{bic.glm()} function in the R package \emph{BMA}. See Raftery et
  al. (2015) and Montgomery and Nyhan (2010). The analysis uses
uniform priors for all independent variables and the only restriction is
that the interaction term and its two component variables are required
to enter or not enter models together. Options were set to be least restrictive.
No Occam's Window was used to narrow the models selected by the initial leap algorithm run over the entire model space and, following Montgomery and Nyhan (pp.~21), the number of best models of each size returned by the leaps algorithm was set to 100,000. In the end, 4,096 models were selected.

Regression results such as those presented above are sometimes sensitive
to minor differences in model specification, such as including or
excluding one variable. Because we do not know the true model, an
idiosyncratic search for optimal model specifications can lead to bias.
One risk is publication bias, if researchers prefer models that confirm
their hypotheses. Another is loss of efficiency if too many unnecessary
variables are included. Finally, an \emph{ad hoc} approach leads to
incomplete representations of model uncertainty (J. M. Montgomery and
Nyhan 2010, 4). An increasingly widespread solution in political science
is Bayesian Model Averaging (BMA), which estimates all possible models
from a set of variables provides posterior probabilities for all
possible coefficients and models.

In particular, a version of the R package \emph{BMA} modified by
Mongtomery and Nyhan was used to search over the entire model space of
Model 2.\footnote{The analysis was conducted using the
  \emph{bic.glmMN()} function, which is a version of the
  \emph{bic.glm()} function in the R package \emph{BMA}. See (Raftery et
  al. 2015) and (J. M. Montgomery and Nyhan 2010)} The analysis uses
uniform priors for all independent variables and the only restriction is
that the interaction term and its two component variables are required
to enter or not enter models together.\footnote{Options were set to be
  least restrictive. No Occam's Window was used to narrow the models
  selected by the initial leap algorithm run over the entire model space
  and, following Montgomery and Nyhan (pp.~21), the number of best
  models of each size returned by the leaps algorithm was set to 100,000}
In the end, 4,096 models were selected.

The results show that the misarchist terms are highly robust to model
selection, with a posterior probability of inclusion equal to 100\% (one
minus the cumulative posterior probability of all models excluding
them). The expected value of the coefficient for \emph{MoralStatism} is
.78 (sd=.29), for \emph{Government} it is -.71 (sd=.26), and for the
interaction term it is -1.43 (sd=.31). \emph{Obama} and \emph{FoxNews}
also had probabilities of inclusion equal to 100\% with expected values
not dissimilar to those estimated in Model 2. \emph{Age} and \emph{Race}
had probabilities of inclusion greater than 50\% but with unstable
signs, as indicated by expected values hardly distinguishable from zero.
All other variables had probabilities of inclusion less than 50\%. More
detailed graphical information is reported in Supplementary Information.

In the main models reported above, we have consciously chosen to include
a large number of control variables because we are presently most
concerned with testing our hypotheses and ruling out rival hypotheses.
BMA provides further evidence that the models presented above likely
contain superfluous independent variables. Though we believe it is best
to rely primarily on the conservative estimates reported above, we
briefly report how our coefficients of interest would change under
different specifications suggested by the BMA. In a model with just
those variables found to be the most robust to model selection
(\emph{Fox News}, \emph{Obama}, \emph{MoralStatism}, and
\emph{Government}), moving from 10th percentile of \emph{MoralStatism}
to 90th within the 90th of \emph{Government}, is associated with the
probablitity of supporting the Tea Party increasing from .01 (sd=.00) to
.50 (sd=.06). Thus, the results from Bayesian Model Averaging suggest
the main results reported above are highly robust and, if anything,
conservative estimates of the partial correlation between misarchism and
Tea Party support.


\begin{figure}[htbp]
\centering
\includegraphics{figures/bma2-1.pdf}
\caption{Inclusion Probabilites from Bayesian Model Averaging}
\end{figure}

\clearpage

\begin{figure}[htbp]
\centering
\includegraphics{figures/bma3-1.pdf}
\caption{Expected Values from Bayesian Model Averaging}
\end{figure}

\clearpage

Multiple Imputation

We use the R package \emph{Amelia} (Honaker, King, and Blackwell 2008) to generate 10
versions of the ANES dataset with missing values imputed and
\emph{Zelig} (Imai, King, and Lau 2009) to obtain pooled regression results via ``Rubin's rules.'' The multiple
imputation algorithm only assumes that missing values are ``missing at
random,'' not necessarily ``missing completely at random.'' In this
context, ``missing at random'' only means that missingness is dependent
on the observed variables. Full numerical results are provided below. Graphical diagnostics for overimputation,
dispersion, and comparing pre- and post-imputation densities for our
main variables suggest no problems or anomolies in the imputation
procedures (see below the numerical results).

Another possible problem is that listwise deletion of all observations
containing missing values may have led to biased estimates, as well as
the simple inefficiency of lost information. In particular, if relative
misarchists were more (or less) likely to respond to certain questions
than other respondents, we may have over-estimated (or under-estimated)
the true partial correlation between our misarchist terms and Tea Party
support. One solution to this problem is multiple imputation of missing
values, which refers to the process of using the information from
observed variables to infer the most likely values for all missing
cells. The process finishes by producing a set of new datasets each of
which samples from the predictive distribution to assign most likely
values to each missing cell. After multiple imputation, the models
discussed above are re-estimated on each imputed dataset and the results
are combined using ``Rubin's rules.'' Specifically, we use the R package
\emph{Amelia} (Honaker, King, and Blackwell 2008) to generate 10
versions of the ANES dataset with missing values imputed and
\emph{Zelig} to obtain pooled regression results. The multiple
imputation algorithm only assumes that missing values are ``missing at
random,'' not necessarily ``missing completely at random.'' In this
context, ``missing at random'' only means that missingness is dependent
on the observed variables.

After pooling the results, the estimates remain substantially the same.
\emph{MoralStatism}, \emph{Government}, and the interaction term remain
signed as in Model 2 with high statistical significance (.98,
p\textless{}.00; -.68, p\textless{}.00; -1.35, p\textless{}.00,
respectively). \emph{FoxNews} and \emph{BornAgain} also remain
substantially the same. Graphical diagnostics for overimputation,
dispersion, and comparing pre- and post-imputation densities for our
main variables suggest no problems or anomolies in the imputation
procedures. For the sake of brevity we have placed the full results
table and further diagnostic information in Supplementary Information.

\begin{table}[ht]
\centering
\begin{tabular}{rrrrr}
  \hline
 & Value & Std. Error & t-stat & p-value \\ 
  \hline
(Intercept) & -2.07 & 0.11 & -18.21 & 0.00 \\ 
  Gender (Female) & -0.10 & 0.09 & -1.21 & 0.23 \\ 
  Income & -0.27 & 0.11 & -2.54 & 0.01 \\ 
  Age & -0.22 & 0.10 & -2.27 & 0.02 \\ 
  Race (White) & -0.29 & 0.11 & -2.77 & 0.01 \\ 
  Education & -0.02 & 0.11 & -0.21 & 0.83 \\ 
  Obama & -0.95 & 0.14 & -6.92 & 0.00 \\ 
  Authoritarianism & 0.10 & 0.10 & 0.97 & 0.33 \\ 
  BornAgain & -0.26 & 0.09 & -2.76 & 0.01 \\ 
  Religion & 0.03 & 0.10 & 0.32 & 0.75 \\ 
  PartyID (Republican) & 0.21 & 0.13 & 1.61 & 0.11 \\ 
  FoxNews & 0.71 & 0.10 & 7.40 & 0.00 \\ 
  Conservatism & 0.98 & 0.13 & 7.46 & 0.00 \\ 
  MoralStatism & 0.07 & 0.20 & 0.35 & 0.73 \\ 
  Government & -0.86 & 0.18 & -4.70 & 0.00 \\ 
  MoralStatism*Government & -0.85 & 0.21 & -3.96 & 0.00 \\ 
   \hline
\end{tabular}
\caption{Pooled Logistic Regression Results From 10 Multiple Imputations} 
\end{table}


\clearpage

\begin{figure}[htbp]
\centering
\includegraphics{figures/missing2-1.pdf}
\caption{Distributions before and after multiple imputation}
\end{figure}

\begin{figure}[htbp]
\centering
\includegraphics{figures/missing3-1.pdf}
\caption{Distributions before and after multiple imputation}
\end{figure}

\clearpage

\begin{figure}[htbp]
\centering
\includegraphics{figures/missing4-1.pdf}
\caption{Diagnostic Plot for Overimputation (1)}
\end{figure}

\begin{figure}[htbp]
\centering
\includegraphics{figures/missing5-1.pdf}
\caption{Diagnostic Plot for Overimputation (2)}
\end{figure}

\clearpage

\begin{figure}[htbp]
\centering
\includegraphics{figures/missing6-1.pdf}
\caption{Diagnostic Plot for Dispersion}
\end{figure}

Matching estimates and sensitivity bounds

Rosenbaum bounds (1988) for a binary dependent variable are calculated using the \emph{binarysens()} function in the R package \emph{rbounds}. See Keele (2014). We generate matching estimates for each of our three independent
variables of interest, one at a time, using one-to-one genetic matching
with replacement. In each case, ``treatment'' is defined as having a
value above the sample mean of the variable of interest. For each
estimate, we balance on all covariates in Model 2 except the two
components of the interaction term and including the treatment
variable's propensity scores with respect to those covariates.

To mitigate the risk of bias from non-random assignment into misarchism
and guard against parametric model dependence, we employ matching to
estimate the effect of misarchism from a subset of highly similar
individuals. We use a genetic matching algorithm to identify that subset
of the original dataset for which the distribution of each covariate is
optimally balanced across both treatment and control groups (Diamond and
Sekhon 2012; Sekhon 2011). In other words, the algorithm obtains the
matched pairs of those ``treated'' and not treated to governmentalism
and moral statism which are otherwise optimally balanced in the
propensity to be treated. The average treatment effect for the treated
obtained from this subset will approximate that which we would obtain
from a randomized experiment, unless some unobserved factor shapes the
propensity to be treated. Although the possibility of omitted variables
can never be ruled out, we can quantify the sensitivity of these
matching estimates to some potential unobserved source of bias. Thus, we
also report sensitivity bounds as developed by Rosenbaum (Rosenbaum
1988).

We generate matching estimates for each of our three independent
variables of interest, one at a time, using one-to-one genetic matching
with replacement. In each case, ``treatment'' is defined as having a
value above the sample mean of the variable of interest. For each
estimate, we balance on all covariates in Model 2 except the two
components of the interaction term and including the treatment
variable's propensity scores with respect to those covariates. We do not
balance on the components of the interaction term, or the interaction
term itself, because this would remove the covariation of
governmentalism and moral statism the effect of which we wish to test,
but we do include the components and the interaction term as covariates.

The average treatment effect on those ``treated'' with greater than the
mean level of moral statism is 0.094, with a standard error of 0.057 and
a p-value of 0.01. The Rosenbaum bounds for this effect suggest that for
it to become statistically insignificant at the 95\% confidence level,
the odds of differential assignment to treatment due to an unobserved
factor would have to be about 2.03.\footnote{Rosenbaum bounds for a
  binary dependent variable reported in this section are calculated
  using the \emph{binarysens()} function in the R package
  \emph{rbounds}. See (Keele 2014)} Thus, the partial correlation
between moral statism and Tea Party support when \emph{Governmentalism}
is set at its mean, as obtained in Model 2, does not appear to be an
artifact of covariate imbalance or parametric modeling assumptions, and
would require a fairly large unobserved source of bias to become
insignificant.

The average treatment effect on those ``treated'' with greater than the
mean level of governmentalism is 0.002, with a standard error of 0.028
and a p-value of 0.88. Therefore the partial correlation previously
estimated between governmentalism and Tea Party support at mean levels
of moral statism appears to have been a spurious result of those with
high values of governmentalism having significantly different values of
some covariate relative to those with low values of governmentalism.

The average treatment effect on those ``treated'' with greater than the
mean level of the interaction term is -0.048, with a standard error of
0.019 and a p-value of 0.01. The Rosenbaum bounds for this effect
suggest that for it to become statistically insignificant at the 95\%
confidence level, the odds of differential assignment to treatment due
to an unobserved factor would have to be about 1.43. Our key
relationship of interest estimated in Model 2 therefore does not appear
generated by systematic assignment into treatment due to any of the
observed covariates and, as with the estimated effect of moral statism,
would require a fairly large unobserved source of bias to become
insignificant.


\clearpage

\subsection{References}

Fabrigar, Leandre R, Duane T Wegener, Robert C MacCallum, and Erin J
Strahan. 1999. ``Evaluating the use of exploratory factor analysis in
psychological research.'' \emph{Psychological Methods} 4(3): 272--99.

Keele, Luke J. 2014. ``rbounds: Perform Rosenbaum bounds sensitivity
tests for matched and unmatched data.''
\url{https://cran.r-project.org/web/packages/rbounds}.

Raftery, Adrian E., Jennifer Hoeting, Chris Volinksy, Ian Painter, and Ka Yee Yeung. 2015. ``BMA: Bayesian Model Averaging.''
\url{https://cran.r-project.org/web/packages/BMA/}.


\end{document}